% Copyright 2021 Edoardo Riggio

% Licensed under the Apache License, Version 2.0 (the "License");
% you may not use this file except in compliance with the License.
% You may obtain a copy of the License at

% 	http://www.apache.org/licenses/LICENSE-2.0

% Unless required by applicable law or agreed to in writing, software
% distributed under the License is distributed on an "AS IS" BASIS,
% WITHOUT WARRANTIES OR CONDITIONS OF ANY KIND, either express or implied.
% See the License for the specific language governing permissions and
% limitations under the License.

\documentclass{article}

\usepackage{hyperref, amsmath, graphicx, amssymb, csquotes, listings}
\usepackage{fancyvrb,newverbs,xcolor}

\graphicspath{ {./assets/} }

\definecolor{cverbbg}{gray}{0.93}

\newenvironment{cverbatim}
 {\SaveVerbatim{cverb}}
 {\endSaveVerbatim
  \flushleft\fboxrule=0pt\fboxsep=.5em
  \colorbox{cverbbg}{\BUseVerbatim{cverb}}%
  \endflushleft
}

\newenvironment{lcverbatim}
 {\SaveVerbatim{cverb}}
 {\endSaveVerbatim
  \flushleft\fboxrule=0pt\fboxsep=.5em
  \colorbox{cverbbg}{%
    \makebox[\dimexpr\linewidth-2\fboxsep][l]{\BUseVerbatim{cverb}}%
  }
  \endflushleft
}

\newcommand{\ctexttt}[1]{\colorbox{cverbbg}{\texttt{#1}}}
\newverbcommand{\cverb}
  {\setbox\verbbox\hbox\bgroup}
  {\egroup\colorbox{cverbbg}{\box\verbbox}}
  
\lstdefinestyle{c++}{
  frame=single, language={C++}, numbers=left, numberstyle=\tiny, tabsize=4, breaklines=true,
  basicstyle=\ttfamily\scriptsize,
  keywordstyle=\color{blue}\ttfamily,
  otherkeywords={WIDTH},
  keywords=[2]{__shared__},
  keywordstyle=[2]\color{orange}\ttfamily,
  stringstyle=\color{red}\ttfamily,
  commentstyle=\color{green}\ttfamily
}

\begin{document}
\begin{titlepage}
    \begin{center}
        \vspace*{1cm}
        
        \Huge
        \textbf{Computer Graphics Formulae}
        
        \vspace{0.5cm}
        \LARGE
        
        \vspace{.5cm}
        
        Edoardo Riggio
   		  \vspace{1.5cm}
       
        \vfill
        
        \today
        
        \vspace{.8cm}
          \Large
          Computer Graphics - SA. 2021 \\
        Computer Science\\
        Universit\`{a} della Svizzera Italiana, Lugano\\
        
    \end{center}
\end{titlepage}

\tableofcontents

\newpage

\section{Vectors}
\subsection{Dot Product}
\vspace{.3cm}
\[ \langle a, b \rangle = \sum^3_{i=1} a_i b_i \]

\subsection{Cross Product}
\vspace{.3cm}
\[ a \times b = \begin{bmatrix} a_2b_3 - a_3b_2 \\ a_3b_1 - a_1b_3 \\ a_1b_2 - a_2b_1 \end{bmatrix} \]

\subsection{Magnitude}
\vspace{.3cm}
\[ |a| = \sqrt{\sum^3_{i=1} a_i^2} \]

\section{Rays and Objects}
\subsection{Ray}
\vspace{.3cm}
\[ \gamma(t) = o + dt \]

\subsection{Ray-Sphere Intersection}
\vspace{.3cm}
\[ a = \langle c, d \rangle \]
\[ D = \sqrt{||~c~||^2 - \langle c, d \rangle^2} \]
\[ t_{1,2} = \langle c, d \rangle \pm \sqrt{r^2 - D^2} \] \\
Where $c$ is the vector from the eye to the center of the sphere, $d$ is the viewing direction, and $r$ is the ray of the circle.

\subsection{Ray-Cone Intersection}
\vspace{.3cm}
\[ a = d_x^2 + d_z^2 - d_y^2 \]
\[ b = 2(d_xo_x - d_zo_z - d_yo_y) \]
\[ c = o_x^2 + o_z^2 - o_y^2 \]
\[ t_{1,2} = \frac{-b \pm \sqrt{b^2 - 4ac}}{2a} \] \\
Where $d$ is the view direction, and $o$ is the position of the eye.

\subsection{Ray-Triangle Intersection}
\vspace{.3cm}
\[ p = \begin{bmatrix} x \\ y \\ z \end{bmatrix} \]
\[ p_i = \begin{bmatrix} x_i \\ y_i \\ z_i \end{bmatrix} \]
\[ n = (p_2 - p_1) \times (p_3 - p1) \]
\[ n_i = (p_{i+1} - p) \times (p_{i-1} - p) \]
\[ W = \frac{||n||}{2} \]
\[ w_i = \frac{||n_i||}{2} \cdot \text{sign}(\langle n_i, n \rangle) \]
\[ \lambda_1(p) = \frac{w_1}{W} ~~~~~~~ \lambda_2(p) = \frac{w_2}{W} ~~~~~~~ \lambda_3(p) = \frac{w_3}{W} \] \\
Where $\lambda_1(p)$, $\lambda_2(p)$ and $\lambda_3(p)$ are the three barycentric coordinates.

\section{Illumination}
\subsection{Diffuse Reflection}
\vspace{.3cm}
\[ I_d = \rho_d \cdot \langle n, l \rangle \cdot I \] \\
Where $\rho_d$ is the diffuse coefficient, $n$ is the normal vector of the point, $l$ is the vector pointing to the light source, and $I$ is the intensity of the light.

\subsection{Ambient Illumination}
\vspace{.3cm}
\[ I_a = \rho_a \cdot I \] \\
Where $\rho_a$ is the ambient coefficient, and $I$ is the ambient light intensity.

\subsection{Specular Reflection}
\vspace{.3cm}
\[ r = 2n \cdot \langle n, l \rangle - l \]
\[ I_s = \rho_s \cdot \langle r, v \rangle^k \cdot I \] \\
Where $n$ is the normal vector of the point, $l$ is the vector pointing to the light source, $\rho_s$ is the reflection coefficient, $r$ is the reflection ray, $v$ is the viewing direction, $k$ is the shininess of the object, and $I$ is the intensity of the light source.

\subsection{Blinn-Phong Specular Reflection}
\vspace{.3cm}
\[ h = \frac{1}{2}(l + v) \]
\[ I_s = \rho_s \cdot \langle n, h \rangle^{4k} \cdot I \] \\
Where $l$ is the vector pointing to the light source, $v$ is the viewing direction, $\rho_s$ is the reflection coefficient, $n$ is the normal of the point, $h$ is the bisection vector, $k$ is the shininess of the object, and $I$ is the intensity of the light source.

\subsection{Distance Attenuation}
\vspace{.3cm}
\[ att(r) = \frac{1}{a_1 + a_2r + a_3r^2} \] \\
Where $r$ is the ray of light, and $a_1$, $a_2$ and $a_3$ are constant values.

\subsection{Phong Lighting Model}
\vspace{.3cm}
\[ I = I_e + \rho_a \cdot I_a + \sum^n_{j = 1} (\rho_d \cdot \langle n, l_j \rangle + \rho_s \cdot \langle n, h_j \rangle^{4k}) \cdot I_j \cdot att(||l_j||) \cdot s_j(p) \] \\
Where $I_e$ is the self-emitting intensity, $\rho_a$ is the diffuse coefficient, $I_a$ is the ambient intensity, $\rho_d$ is the diffuse coefficient, $n$ is the normal vector, $l$ is the vector pointing to the light source, $\rho_s$ is the specular coefficient, $h$ is the bisection vector, $v$ is the direction vector, $k$ is the shininess, $I_j$ is the intensity of the $j$-th light source, and $s_j(p)$ indicates if the point $p$ indicates whether the point is in shadow or not -- 1 $p$ is not in shadow, 0 $p$ is in shadow.

\subsection{Gamma Correction and Tone Mapping}
\vspace{.3cm}
\[ I_{in} = \max((\alpha \cdot I^\beta)^{\frac{1}{\gamma}}, 1.0) \] \\
Where $I_{in}$ is the input for our display, $I$ is the intensity computed by the Phong model, $\gamma$ is the gamma coefficient, and $\alpha$ and $\beta$ are the tone mapping coefficients.

\section{Transformations}
\subsection{Translation}
\vspace{.3cm}
\[ T = \begin{bmatrix} 1 & 0 & 0 & t_x \\ 0 & 1 & 0 & t_y \\ 0 & 0 & 1 & t_z \\ 0 & 0 & 0 & 1 \end{bmatrix} \]

\subsection{Shear}
\vspace{.3cm}
\[ S_{yz} = \begin{bmatrix} 1 & 0 & 0 & 0 \\ d_y & 1 & 0 & 0 \\ d_z & 0 & 1 & 0 \\ 0 & 0 & 0 & 1 \end{bmatrix} ~~~~~ S_{xz} = \begin{bmatrix} 1 & d_x & 0 & 0 \\ 0 & 1 & 0 & 0 \\ 0 & d_z & 1 & 0 \\ 0 & 0 & 0 & 1 \end{bmatrix} \]
\[ S_{xy} = \begin{bmatrix} 1 & 0 & d_x & 0 \\ 0 & 1 & d_y & 0 \\ 0 & 0 & 1 & 0 \\ 0 & 0 & 0 & 1 \end{bmatrix} \]

\subsection{Scale}
\vspace{.3cm}
\[ S = \begin{bmatrix} s_x & 0 & 0 & 0 \\ 0 & s_y & 0 & 0 \\ 0 & 0 & s_z & 0 \\ 0 & 0 & 0 & 1 \end{bmatrix} \]

\subsection{Rotation}
\vspace{.3cm}
\[ R_x = \begin{bmatrix} 1 & 0 & 0 & 0 \\ 0 & \cos \alpha & -\sin \alpha & 0 \\ 0 & \sin \alpha & \cos \alpha & 0 \\ 0 & 0 & 0 & 1 \end{bmatrix} ~~~~~ R_y = \begin{bmatrix} \cos \alpha & 0 & \sin \alpha & 0 \\ 0 & 1 & 0 & 0 \\ -\sin \alpha & 0 & \cos \alpha & 0 \\ 0 & 0 & 0 & 1 \end{bmatrix} \]
\[ R_z = \begin{bmatrix} \cos \alpha & -\sin \alpha & 0 & 0 \\ \sin \alpha & \cos \alpha & 0 & 0 \\ 0 & 0 & 1 & 0 \\ 0 & 0 & 0 & 1 \end{bmatrix} \]

\subsection{Normal}
\vspace{.3cm}
\[ n_{new} = Mn \]
\[ n_{new} = (M^{-1})^Tn \] \\
Where $M$ is the transformation matrix. The first equation is used in the case that the transformation preserves angles, otherwise we use the second equations.

\section{Advanced}
\subsection{Snell's Law}
\vspace{.3cm}
\[ \frac{\sin \Theta_1}{\sin \Theta_2} = \frac{\delta_1}{\delta_2} = \frac{v_1}{v_2} \] \\
Where $\delta$ is the index of refraction of the medium, and $v$ is the velocity of light in the medium.

\subsection{Reflection}
\vspace{.3cm}
\[ r = i - 2n \cdot \langle n, i \rangle \] \\
Where $r$ is the reflection vector, $i$ is the viewing direction, and $n$ is the normal of the point.

\subsection{Refraction}
\vspace{.3cm}
\[ a = n \cdot \langle n, i \rangle \]
\[ b = i - a \]
\[ \beta = \frac{\delta_1}{\delta_2} \]
\[ \alpha = \sqrt{1 + ( 1 - \beta^2) \frac{||b||^2}{||a||^2}} \]
\[ r = \alpha a + \beta b \] \\
Where $n$ is the normal of the vector, $i$ is the viewing direction, and $\delta$ is the index of refraction of the medium.

\subsection{Fresnel Effect}
\vspace{.3cm}
\[ F_{\text{Rl}} = \frac{1}{2} \left( \left( \frac{\delta_2 \cos\Theta_1 - \delta_1 \cos \Theta_2}{\delta_2 \cos\Theta_1 + \delta_1 \cos \Theta_2} \right)^2 + \left( \frac{\delta_1 \cos\Theta_1 - \delta_2 \cos \Theta_2}{\delta_1 \cos\Theta_1 + \delta_2 \cos \Theta_2} \right)^2 \right) \]
\[ F_{\text{Rf}} = 1 - F_{\text{Rl}} \]

\section{Transformation Pipeline}
\subsection{Vectors}
\vspace{.3cm}
\[ eye = VP \]
\[ z' = \frac{VPN}{||VPN||} \]
\[ x' = \frac{VUP \times z'}{||VUP|| \times z'||} \]
\[ y' = z' \times x' \] \\
Where $VP$ is the camera position, $VPN$ is the view plane normal, and $VUP$ is the view up vector

\subsection{Viewing Matrix}
\vspace{.3cm}
\[ VM = \begin{bmatrix} - & x^T & - & -x'^T \cdot eye \\ - & y^T & - & -y'^T \cdot eye \\ - & z^T & - & -z'^T \cdot eye \\ 0 & 0 & 0 & 1 \end{bmatrix} \]

\subsection{Projection Matrix}
\vspace{.3cm}
\[ PM = \begin{bmatrix} \frac{1}{r} & 0 & 0 & 0 \\ 0 & \frac{1}{t} & 0 & 0 \\ 0 & 0 & \frac{-2}{f-n} & -\frac{f+n}{f-n} \\ 0 & 0 & 0 & 1 \end{bmatrix} \]

\section{Texture Pipeline}
\subsection{Normal Mapping}
\vspace{.3cm}
\[ n_{new} = \begin{bmatrix} n_x \\ n_y \\ n_z \end{bmatrix} = 2 \begin{bmatrix} R \\ G \\ B \end{bmatrix} - \begin{bmatrix} 1 \\ 1 \\ 1 \end{bmatrix} \]

\subsection{Tangent Space}
\vspace{.3cm}
\[ p_i - p_j = \langle (u_i - u_j), T \rangle + \langle (v_i - v_j), B \rangle \]
\[ T = T - \langle N, T \rangle \cdot N \]
\[ B = N \times T \] \\
Where $T$ is the tangent, $B$ is the bitangent, and $N$ is the normal to the point.

\end{document}