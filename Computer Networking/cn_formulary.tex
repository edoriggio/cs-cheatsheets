% Copyright 2021 Edoardo Riggio

% Licensed under the Apache License, Version 2.0 (the "License");
% you may not use this file except in compliance with the License.
% You may obtain a copy of the License at

% 	http://www.apache.org/licenses/LICENSE-2.0

% Unless required by applicable law or agreed to in writing, software
% distributed under the License is distributed on an "AS IS" BASIS,
% WITHOUT WARRANTIES OR CONDITIONS OF ANY KIND, either express or implied.
% See the License for the specific language governing permissions and
% limitations under the License.

\documentclass{article}

\usepackage{amsmath, graphicx, hyperref}
\usepackage{fancyvrb, newverbs, xcolor}

\definecolor{cverbbg}{gray}{0.93}

\newenvironment{cverbatim}
 {\SaveVerbatim{cverb}}
 {\endSaveVerbatim
  \flushleft\fboxrule=0pt\fboxsep=.5em
  \colorbox{cverbbg}{\BUseVerbatim{cverb}}%
  \endflushleft
}

\newenvironment{lcverbatim}
 {\SaveVerbatim{cverb}}
 {\endSaveVerbatim
  \flushleft\fboxrule=0pt\fboxsep=.5em
  \colorbox{cverbbg}{%
    \makebox[\dimexpr\linewidth-2\fboxsep][l]{\BUseVerbatim{cverb}}%
  }
  \endflushleft
}

\newcommand{\ctexttt}[1]{\colorbox{cverbbg}{\texttt{#1}}}
\newverbcommand{\cverb}
  {\setbox\verbbox\hbox\bgroup}
  {\egroup\colorbox{cverbbg}{\box\verbbox}}

\begin{document}
\begin{titlepage}
    \begin{center}
        \vspace*{1cm}
        
        \Huge
        \textbf{Computer Networking Formulary}
        
        \vspace{0.5cm}
        \LARGE
        
        \vspace{.5cm}
        
        Edoardo Riggio
   		  \vspace{1.5cm}
       
        \vfill
        
        \today
        
        \vspace{.8cm}
          \Large
          Computer Networking - SP. 2021 \\
        Computer Science\\
        Universit\`{a} della Svizzera Italiana, Lugano\\
        
    \end{center}
\end{titlepage}

\tableofcontents

\newpage

\section{Computer Networking and the Internet}
\subsection{Queuing Delay}
\vspace{.3cm}
\[ d_{queue} = \frac{L \cdot a}{R} \cdot \frac{L}{R} \cdot \left( 1 - \frac{L \cdot a}{R} \right) \] \\
Where \textbf{L} is the length of the packet, \textbf{a} is the average rate of packets per second, and \textbf{R} is the transmission rate of the link.

\subsection{Transmission Delay}
\vspace{.3cm}
\[ d_{transmission} = \frac{L}{R} \] \\
Where \textbf{L} is the length of the packet, and \textbf{R} is the transmission rate of the link.

\subsection{Propagation Delay}
\vspace{.3cm}
\[ d_{propagation} = \frac{d}{s} \] \\
Where \textbf{d} is the length of the link, and \textbf{s} is the propagation speed of the link.

\subsection{Nodal Delay}
\vspace{.3cm}
\[ d_{nodal} = d_{processing} + d_{queue} + d_{transmission} + d_{propagation} \]

\subsection{End-to-End Delay}
\vspace{.3cm}
\[ d_{end-to-end} = N \cdot (d_{processing} + d_{queue} + d_{transmission} + d_{propagation}) \] \\
Where \textbf{N} is the number of nodes. This can be used if all nodes have the same nodal delay.
\[ d_{end-to-end} = \sum^N_1 d_{nodal~i} \] \\
Where \textbf{N} is the number of nodes, and $d_{nodal~i}$ is the nodal delay of the $i$th node.

\subsection{End-to-End Throughput}
\vspace{.3cm}
\[ t_{end-to-end} = \frac{F}{T} \] \\
Where \textbf{F} is the length of the packet, and \textbf{T} is the transfer rate of the link.

\section{Application Layer}
\subsection{Client-Server File Distribution}
\vspace{.3cm}
\[ D_{CS} = \max \left\{ \frac{NF}{u_s}, \frac{F}{d_{min}} \right\} \] \\
Where \textbf{F} is the length of the packet, \textbf{d} is the length of a link, \textbf{N} is the number of clients, $u_s$ is the server's upload rate.

\subsection{P2P File Distribution}
\vspace{.3cm}
\[ D_{P2P} = \max \left\{ \frac{F}{u_s}, \frac{F}{d_{min}}, \frac{NF}{u_s + \sum^N_{i=1}u_i} \right\} \] \\
Where \textbf{F} is the length of the packet, \textbf{d} is the length of a link, \textbf{N} is the number of peers, $u_s$ is the server's upload rate, and $u$ is the peer's upload rate.

\section{Transport Layer}
\subsection{Estimating the RTT}
\vspace{.3cm}
\[ RTT_{est} = (1 - \alpha) \cdot RTT_{est} + \alpha \cdot RTT_{sample} \] \\
Where $RTT_{est}$ is the currently computed estimate, and $RTT_{sample}$ is the RTT measurement.

\subsection{Estimating the Timeout}
\vspace{.3cm}
\[ T = RTT_{est} + 4 \cdot ((1-\beta) \cdot RTT_{dev} + \beta \cdot | RTT_{sample} - RTT_{est} |) \] \\
Where $RTT_{est}$ is the currently computed estimate, and $RTT_{sample}$ is the RTT measurement.

\section{Network Layer}
\subsection{Output Port Queuing}
\vspace{.3cm}
\[ Buff = RTT \cdot \frac{C}{N} \] \\
Where \textbf{C} is the link capacity, and \textbf{N} is the number of packets.

\subsection{Weighted Fair Queuing}
\vspace{.3cm}
\[ W_a = \frac{W_i}{\sum_i i} \] \\
Where \textbf{W} is the weight of the $i$th packet, and \textbf{i} is the number of the packet.

\section{Link Layer}
\subsection{Bit Errors}
\vspace{.3cm}
\[ BER = \frac{N_{err}}{N_{tot}} \] \\
Where $N_{err}$ is the number of bit errors, and $N_{tot}$ is the number of bit errors.

\subsection{Cyclic Redundancy Check}
\vspace{.3cm}
\[ R = (D \cdot 2^r) \% (G) \] \\
Where \textbf{D} is a piece of data where a CRC field is added, \textbf{r} is the length of the CRC field, and \textbf{G} is the generator -- bit pattern of $r + 1$ bits.

\section{Wireless Networks}
\subsection{Power Difference}
\vspace{.3cm}
\[ P_{rx}(d) \approx \beta \cdot P_{tx} \cdot \frac{1}{d^\alpha} \] \\
Where $P_{rx}$ is the received power, $\beta$ is the multiplicative attenuation factor, $P_{tx}$ is the transmitted power, \textbf{d} is the distance between transmitter and receiver, and $\alpha$ is the path loss exponent.

\subsection{Received Signal Strength Indicator}
\vspace{.3cm}
\[ RSSI = 10~log_{10}\left(\frac{P_{rx}}{1~mW}\right) \] \\
Where $P_{rx}$ is the received power


\end{document}




















